\documentclass{sigchi}

% Use this command to override the default ACM copyright statement (e.g. for preprints). 
% Consult the conference website for the camera-ready copyright statement.
\toappear{}



% Arabic page numbers for submission. 
% Remove this line to eliminate page numbers for the camera ready copy
\pagenumbering{arabic}


% Load basic packages
\usepackage{balance}  % to better equalize the last page
\usepackage{graphics} % for EPS, load graphicx instead
\usepackage{times}    % comment if you want LaTeX's default font
\usepackage{url}      % llt: nicely formatted URLs

% llt: Define a global style for URLs, rather that the default one
\makeatletter
\def\url@leostyle{%
  \@ifundefined{selectfont}{\def\UrlFont{\sf}}{\def\UrlFont{\small\bf\ttfamily}}}
\makeatother
\urlstyle{leo}


% To make various LaTeX processors do the right thing with page size.
\def\pprw{8.5in}
\def\pprh{11in}
\special{papersize=\pprw,\pprh}
\setlength{\paperwidth}{\pprw}
\setlength{\paperheight}{\pprh}
\setlength{\pdfpagewidth}{\pprw}
\setlength{\pdfpageheight}{\pprh}

% Make sure hyperref comes last of your loaded packages, 
% to give it a fighting chance of not being over-written, 
% since its job is to redefine many LaTeX commands.
\usepackage[pdftex]{hyperref}
\hypersetup{
pdftitle={SIGCHI Conference Proceedings Format},
pdfauthor={LaTeX},
pdfkeywords={SIGCHI, proceedings, archival format},
bookmarksnumbered,
pdfstartview={FitH},
colorlinks,
citecolor=black,
filecolor=black,
linkcolor=black,
urlcolor=black,
breaklinks=true,
}

% create a shortcut to typeset table headings
\newcommand\tabhead[1]{\small\textbf{#1}}


% End of preamble. Here it comes the document.
\begin{document}

\title{Campus Sherpa}

\numberofauthors{4}
\author{
  \alignauthor Christopher Ford\\
    \email{cjford@mit.edu}\\
  \alignauthor Ganesh Ajjanagadde\\
    \email{gajjang@mit.edu}\\  
  \alignauthor Harihar Subramanyam\\
    \email{hsubrama@mit.edu}\\
   \alignauthor James Thomas\\
    \email{jjthomas@mit.edu}\\
}

\maketitle

\begin{abstract}
In this paper, we outline the development and motivation for Campus Sherpa, a mobile application that allows users to make and take custom tours of the MIT campus. Our initial field study indicated that the tourist experience at MIT left something to be desired. Our field evaluation, conducted after the development of our application, showed us that our Campus Sherpa was perfectly suited to improve upon the tour taking experience at MIT. (TODO: INCORPORATE FIELD STUDY RESULTS HERE)

\end{abstract}

\keywords{
	Geolocations; campus; tours; location tracking; tourism; campus; sharing
}
% Stubbed out these sections for now, don't know what to do with them

%\keywords{
%	Guides; instructions; author's kit; conference publications;
%	keywords should be separated by a semi-colon.
%	\textcolor{red}{Mandatory section to be included in your final version.}
%}
%
%\category{H.5.m.}{Information Interfaces and Presentation (e.g. HCI)}{Miscellaneous}
%
%See: \url{http://www.acm.org/about/class/1998/}
%for more information and the full list of ACM classifiers
%and descriptors. 
%\textcolor{red}{Mandatory section to be included in your
%final version. On the submission page only the classifiers'
%letter-number combination will need to be entered.}

\section{Introduction}

Our domain of interest is tourism, more specifically, tourism at MIT. The MIT campus is filled with plenty of sights to see and plenty of people to visit them. The goal of this application is to pair people with the places they want to see.

MIT receives a number of visitors each day from various demographics - prospective undergraduates/graduates, tourists visiting Boston/Cambridge, professionals or researchers attending a conference on campus, middle school students enrolled in Splash, etc. Many of these visitors will take a campus tour which will take them to the main sights at MIT (ex. the Student Center, gymnasium, auditorium, main hallway, and architecturally impressive computer science laboratory). However, this �one-size-fits-all� nature of the tour fails to take into account the interests of each visitor and show them all the sights that they would enjoy seeing. Primary research questions include: ``How can we fix this with an easy to use mobile application?", ``Is a mobile application the right way to solve this problem?", and ``How can we take advantage of modern cell phone technologies to make the use of our application as easy as possible?"

The goal of this application is to allow users to pursue a tour (likely after the main campus tour) which will help them make the most of their time at MIT. For instance, a prospective graduate student in Biology would follow a tour of the biology labs while a tourist interested in history would be taken to the most historically important sites at MIT. 

Since smartphones offer location awareness and mobility, they are the obvious platform choice for this application. By deploying this application for iOS, users can take out their smartphone when they arrive at MIT, select a tour based on their interest, and then begin exploring sites that will be most useful to them. 

The second aspect of this application is to allow users to chronicle their tours and share them with others. As prospective students compare MIT to other schools or as tourists reminisce about their visit to MIT, the ability to record their tours (by associating locations with text, photos, videos, and links to related content) will aid them in remembering. For current students, this offers a way to record the precious memories they make at MIT and share them with others.

Thus, the primary audience will be people who aim to tour MIT and record their experience there. The secondary audience will be current MIT students who would like to chronicle their lives at MIT. While some MIT tours will be pre-installed, the majority should be user-created. 

Based on our own experiences and the results of our interviews, people have varied tastes and desired destinations when visiting MIT, and we aim to make sure that their trip to MIT is as enjoyable and productive as possible. By creating an application to let users create, share, and follow custom tours of MIT, we hope to make college tours custom fit to each person�s taste, to make memories of MIT easy to record and share, and to make a platform for sharing tours.

\section{Related Works}

There exist a number of products and applications aimed at making tour taking more exciting and enjoyable. Interactive exhibits at museums and college campuses aim to improve upon the tour taking experience by allowing tour takers to interact with what they are touring. These interactive kiosks often serve many tourists at once, and fail to provide a custom tailored experience to each user. Often, these interactive stations can get bogged down with traffic, resulting in long wait times which only frustrates the tourist.

There exist a few mobile applications that try to enrich tourism as well. Applications in this domain fall into two main categories: applications that try to replace the tour taking experience all together, and applications that try to supplement existing tours. Applications in the first category include the Library of Congress Virtual Tour, Canadian Museum of Civilization, and American Museum of Natural History mobile applications. Although these virtual tours allow more users to "explore" a tourist destination, they fail to replicate the experience of touring in person. A 3.5" to 5" screen is not a good way to replicate a tour.

Applications that try to supplement existing tours include: the Boston Freedom Tail, Walking Cinema: Murder on Beacon Hill, and LAT Star Walk mobile applications. These applications do take in the users position so they can serve them content when they reach specific locations, but they lack the ability to create and share custom tours made by other users. With these applications, users are limited to the content made by the developers. While these applications usually only provide users with information specific to one tour, Campus Sherpa allows users to browse and take many tours of MIT, as created by other users.

TODO: ADD ACADEMIC RELATED WORKS

\section{Background}

(TODO: EXPAND SECTION)
Motivated by the above observation, we first identified the types of people touring MIT. We decided that for the purposes of campus tours, we can broadly classify people as current students, prospective freshmen, parents, and tourists. In order to gather information about these groups, we conducted a survey of these four groups, with some questions common to all four groups and some tailored to a specific group. Here are some of the questions we asked (TODO: SHORTEN QUESTION LIST IF NEEDED):

\textbf{Everyone}:

\begin{itemize}
	\item What is the most memorable site you’ve visited here? Why?
	\item What is the one thing you most wish to see here? Why?
	\item Have you gone on a tour of MIT?
	\begin{enumerate}
		\item Are there any questions the tour guide didn’t answer?
		\item Did the tour take you everywhere you wanted?
		\item Did the tour leave out any place you wanted to see?
	\end{enumerate}
\end{itemize}

\textbf{Current Students}:
\begin{itemize}
	\item What are three things you looked for when touring MIT?
	\item What is one place that the tour didn’t cover, which you think tourists should see?
\end{itemize}

\textbf{Prospective Freshman}:
\begin{itemize}
	\item If you could ask a current student one question about MIT, what would it be?
	\item What aspect of MIT do you want explore the most?
\end{itemize}

\textbf{Parents}:
\begin{itemize}
	\item What are three places you want to tour at MIT that you think your child might not want to?
	\item If you could ask a current student one question about MIT, what would it be?
\end{itemize}

\textbf{Tourists}:
\begin{itemize}
	\item How long will you be visiting MIT?
	\item In a sentence, why did you want to come visit MIT?
\end{itemize}

Approximately twenty people responded to the survey, evenly distributed across the four demographics. Some of the responses to the survey are:

``There are always too many things to see \ldots I am wishing I can see more later'' (current student)

``Campus Preview Weekend (CPW) was really rushed, and I did not get to see everything'' (prospective freshman)

``I wish I could get a better sense of MIT culture. The tour guide briefly touched on how each dorm was different, but did not go into detail. '' (prospective freshman)

``I wish they had shown us where the students hang out '' (parent)

``I wish I could get a better sense of MIT culture. The tour guide briefly touched on how each dorm is different, but didn't go into detail. '' (tourist)

From the above responses, we saw that a common complaint was that people were not able to visit places they wanted to with the regular campus tours. In order to solve this problem, we proposed an application that helps create and share custom tours of the MIT campus called ``Campus Sherpa''.

Since smartphones offer location awareness and mobility, they are the obvious platform choice for ``Campus Sherpa''. By deploying this application for iOS, users can take out their smartphone when they arrive at MIT, select a tour based on their interest, and then begin exploring sites that will be most useful to them.

The second aspect of ``Campus Sherpa'' that we wished to support is to allow users to chronicle their tours and share them with others. As prospective students compare MIT to other schools or as tourists reminisce about their visit to MIT, the ability to record their tours (by associating locations with text, photos, videos, and links to related content) will aid them in remembering. For current students, this offers a way to record the precious memories they make at MIT and share them with others. Thus, the primary audience will be people who aim to tour MIT and record their experience there. The secondary audience will be current MIT students who would like to chronicle their lives at MIT.

While some MIT tours will be pre-installed, the majority should be user-created.
Based on our own experiences and the responses to our survey, people have varied tastes and desired destinations when visiting MIT, and we aimed to make sure that their trip to MIT is as enjoyable and productive as possible.

By creating an application to let users create, share, and follow custom tours of MIT, we hope to make college tours custom fit to each person’s taste, to make memories of MIT easy to record and share, and to make a platform for sharing tours.

\section{System Description}

``Campus Sherpa'' allows users to achieve two goals: 

\begin{itemize}
	\item Take custom tours
	\item Create tours based on experiences.
\end{itemize}

When users launch the application, they are presented with a screen (Figure \ref{browser}) that lists all tours that have been created (by any user). If they would like to take a tour, they can select one from the list, and they will be directed to a "Start Tour" page (Figure \ref{start-tour}), which includes a map of all of the locations on the tour. If any of the location pins are clicked, the name of the corresponding location is shown.

Once the user starts a tour, they are presented with a screen for the first location. The screen contains a picture of the location, two buttons for navigating to the previous location and the next location, another button that brings up a map with all of the locations (the same map from Figure \ref{start-tour}), and a final button that shows a list of media items associated with the current location (Figure \ref{take-tour-media}).

\subsection{Technical Details}
We use Parse for the backend. Parse is a backend-as-a-service which provides an SDK for iOS which allows users to persist objects on the server. This is much simpler than building a backend from scratch. For this project, we persist and retrieve latitude/longitude, text, pictures, and audio - which Parse supports. Since this application does not run in a browser, we do not have any browser dependencies (TODO: EXPAND THIS)
    
\section{Field Study}

For our field study, we decided to combine quantitative and qualitative analysis of Campus Sherpa's user experience. For each user, we tracked which screens were viewed most often, which buttons were pressed most frequently, and for how much time a particular screen was viewed for. This gave us a sense of what our users were interacting with, or getting stuck on, from a quantitative perspective.

We also kept a detailed log of how we observed our users interacting with Campus Sherpa. We observed and noted body language and facial expression and did our best to map both positive and negative interactions with our application to specific screens.

Additionally, we surveyed users after they tried our application to get some of their feedback as well. User feedback, combined with qualitative observations, and quantitative data gave us a detailed look into how well Campus Sherpa worked for our users.


TODO: ADD RESULTS OF FIELD STUDY

\section{Discussion}

Prepare your submissions on a word processor or typesetter.  Please
note that page layout may change slightly depending upon the printer
you have specified.  \LaTeX\ sometimes will create overfull lines
that extend into columns.  To attempt to combat this, the .cls
file has a command, {\textbackslash}sloppy, that essentially asks
\LaTeX\ to prefer underfull lines with extra whitespace.  For more
details on this, and info on how to control it more finely, check out
{\url{http://www.economics.utoronto.ca/osborne/latex/PMAKEUP.HTM}}.

\subsection{Future Work}

TODO: ADD FUTURE WORK

\subsection{Conslusion}

TODO: ADD CONCLUSION

\section{Acknowledgments}

We thank the participants of our studies and exercises for their time and effort, as well as for allowing us
to use their thoughts and comments in both this paper, and in the development process of Campus Sherpa.

We also thank Ed Barrett, Frank Bentley, and Jason Martin Lipshin and the entire 21W.789 staff for their continued support and advice. Without them, Campus Sherpa would not have been possible.

% Balancing columns in a ref list is a bit of a pain because you
% either use a hack like flushend or balance, or manually insert
% a column break.  http://www.tex.ac.uk/cgi-bin/texfaq2html?label=balance
% multicols doesn't work because we're already in two-column mode,
% and flushend isn't awesome, so I choose balance.  See this
% for more info: http://cs.brown.edu/system/software/latex/doc/balance.pdf
%
% Note that in a perfect world balance wants to be in the first
% column of the last page.
%
% If balance doesn't work for you, you can remove that and
% hard-code a column break into the bbl file right before you
% submit:
%
% http://stackoverflow.com/questions/2149854/how-to-manually-equalize-columns-
% in-an-ieee-paper-if-using-bibtex
%
% Or, just remove \balance and give up on balancing the last page.
%
\balance

\section{References}

\bibliographystyle{acm-sigchi}
\bibliography{testbib}
\end{document}
