\documentclass{sigchi}

% Use this command to override the default ACM copyright statement (e.g. for preprints). 
% Consult the conference website for the camera-ready copyright statement.


%% EXAMPLE BEGIN -- HOW TO OVERRIDE THE DEFAULT COPYRIGHT STRIP -- (July 22, 2013 - Paul Baumann)
% \toappear{Permission to make digital or hard copies of all or part of this work for personal or classroom use is 	granted without fee provided that copies are not made or distributed for profit or commercial advantage and that copies bear this notice and the full citation on the first page. Copyrights for components of this work owned by others than ACM must be honored. Abstracting with credit is permitted. To copy otherwise, or republish, to post on servers or to redistribute to lists, requires prior specific permission and/or a fee. Request permissions from permissions@acm.org. \\
% {\emph{CHI'14}}, April 26--May 1, 2014, Toronto, Canada. \\
% Copyright \copyright~2014 ACM ISBN/14/04...\$15.00. \\
% DOI string from ACM form confirmation}
%% EXAMPLE END -- HOW TO OVERRIDE THE DEFAULT COPYRIGHT STRIP -- (July 22, 2013 - Paul Baumann)


% Arabic page numbers for submission. 
% Remove this line to eliminate page numbers for the camera ready copy
\pagenumbering{arabic}


% Load basic packages
\usepackage{balance}  % to better equalize the last page
\usepackage{graphics} % for EPS, load graphicx instead
\usepackage{times}    % comment if you want LaTeX's default font
\usepackage{url}      % llt: nicely formatted URLs

% llt: Define a global style for URLs, rather that the default one
\makeatletter
\def\url@leostyle{%
  \@ifundefined{selectfont}{\def\UrlFont{\sf}}{\def\UrlFont{\small\bf\ttfamily}}}
\makeatother
\urlstyle{leo}


% To make various LaTeX processors do the right thing with page size.
\def\pprw{8.5in}
\def\pprh{11in}
\special{papersize=\pprw,\pprh}
\setlength{\paperwidth}{\pprw}
\setlength{\paperheight}{\pprh}
\setlength{\pdfpagewidth}{\pprw}
\setlength{\pdfpageheight}{\pprh}

% Make sure hyperref comes last of your loaded packages, 
% to give it a fighting chance of not being over-written, 
% since its job is to redefine many LaTeX commands.
\usepackage[pdftex]{hyperref}
\hypersetup{
pdftitle={SIGCHI Conference Proceedings Format},
pdfauthor={LaTeX},
pdfkeywords={SIGCHI, proceedings, archival format},
bookmarksnumbered,
pdfstartview={FitH},
colorlinks,
citecolor=black,
filecolor=black,
linkcolor=black,
urlcolor=black,
breaklinks=true,
}

% create a shortcut to typeset table headings
\newcommand\tabhead[1]{\small\textbf{#1}}


% End of preamble. Here it comes the document.
\begin{document}

\title{Campus Sherpa}

\numberofauthors{4}
\author{
  \alignauthor Christopher Ford\\
    \email{cjford@mit.edu}\\
    \affaddr{(650) 784-6891}
  \alignauthor Ganesh Ajjanagadde\\
    \email{gajjang@mit.edu}\\
    \affaddr{(510) 358-5239}    
  \alignauthor Harihar Subramanyam\\
    \email{hsubrama@mit.edu}\\
    \affaddr{(508) 733-3906}
   \alignauthor James Thomas\\
    \email{jjthomas@mit.edu}\\
    \affaddr{(408) 837-1037}
}

\maketitle

\begin{abstract}
We describe the motivation, design, and evaluation of the Campus Sherpa app, which helps users create and take custom tours of the Massachusetts Institute of Technology (MIT) campus. The motivation for the app came from a study where we found a large number of people, such as prospective freshmen, incoming graduate students, tourists, and parents all expressing the need for custom tours of the campus. Our app was designed to address this need by allowing users to create, upload, and take custom tours of the MIT campus. (TODO: INCORPORATE FIELD STUDY RESULTS HERE)

\end{abstract}

\keywords{
	Geolocations; campus; tours; location tracking; tourism; campus; sharing
}
% Stubbed out these sections for now, don't know what to do with them

%\keywords{
%	Guides; instructions; author's kit; conference publications;
%	keywords should be separated by a semi-colon.
%	\textcolor{red}{Mandatory section to be included in your final version.}
%}
%
%\category{H.5.m.}{Information Interfaces and Presentation (e.g. HCI)}{Miscellaneous}
%
%See: \url{http://www.acm.org/about/class/1998/}
%for more information and the full list of ACM classifiers
%and descriptors. 
%\textcolor{red}{Mandatory section to be included in your
%final version. On the submission page only the classifiers'
%letter-number combination will need to be entered.}

\section{Introduction}

MIT receives a number of visitors each day from various demographics- prospective
undergraduates/graduates, tourists visiting Boston/Cambridge, professionals or researchers
attending a conference on campus, middle school students enrolled in Splash,  etc. Many of
these visitors take campus tours which take them to the main sights at MIT, such as the Student Center, gymnasium, auditorium, main hallway, and architecturally impressive computer science laboratory). However, this ``one size fits all'' nature of the tour fails to take into account the interests of each visitor and show them all the sights that they would enjoy seeing. For instance, it is plausible that a prospective graduate student in biology would like to follow a tour of the biology labs, while a tourist interested in history would like to be taken to the most historically important sites at MIT. 

\section{Related Works}

There exist a number of products and applications aimed at making tour taking more exciting and enjoyable. Interactive exhibits at museums and college campuses aim to improve upon the tour taking experience by allowing tour takers to interact with what they are touring. These interactive kiosks often serve many tourists at once, and fail to provide a custom tailored experience to each user. Often, these interactive stations can get bogged down with traffic, resulting in long wait times which only frustrates the tourist.

There exist a few mobile applications that try to enrich tourism as well. Applications in this domain fall into two main categories: applications that try to replace the tour taking experience all together, and applications that try to supplement existing tours. Applications in the first category include the Library of Congress Virtual Tour, Canadian Museum of Civilization, and American Museum of Natural History mobile applications. Although these virtual tours allow more users to "explore" a tourist destination, they fail to replicate the experience of touring in person. A 3.5" to 5" screen is not a good way to replicate a tour.

Applications that try to supplement existing tours include: the Boston Freedom Tail, Walking Cinema: Murder on Beacon Hill, and LAT Star Walk mobile applications. These applications do take in the users position so they can serve them content when they reach specific locations, but they lack the ability to create and share custom tours made by other users. With these applications, users are limited to the content made by the developers. While these applications usually only provide users with information specific to one tour, Campus Sherpa allows users to browse and take many tours of MIT, as created by other users.

\section{Background}

(TODO: EXPAND SECTION)
Motivated by the above observation, we first identified the types of people touring MIT. We decided that for the purposes of campus tours, we can broadly classify people as current students, prospective freshmen, parents, and tourists. In order to gather information about these groups, we conducted a survey of these four groups, with some questions common to all four groups and some tailored to a specific group. Here are some of the questions we asked (TODO: SHORTEN QUESTION LIST IF NEEDED):

\textbf{Everyone}:

\begin{itemize}
	\item What is the most memorable site you’ve visited here? Why?
	\item What is the one thing you most wish to see here? Why?
	\item Have you gone on a tour of MIT?
	\begin{enumerate}
		\item Are there any questions the tour guide didn’t answer?
		\item Did the tour take you everywhere you wanted?
		\item Did the tour leave out any place you wanted to see?
	\end{enumerate}
\end{itemize}

\textbf{Current Students}:
\begin{itemize}
	\item What are three things you looked for when touring MIT?
	\item What is one place that the tour didn’t cover, which you think tourists should see?
\end{itemize}

\textbf{Prospective Freshman}:
\begin{itemize}
	\item If you could ask a current student one question about MIT, what would it be?
	\item What aspect of MIT do you want explore the most?
\end{itemize}

\textbf{Parents}:
\begin{itemize}
	\item What are three places you want to tour at MIT that you think your child might not want to?
	\item If you could ask a current student one question about MIT, what would it be?
\end{itemize}

\textbf{Tourists}:
\begin{itemize}
	\item How long will you be visiting MIT?
	\item In a sentence, why did you want to come visit MIT?
\end{itemize}

Approximately twenty people responded to the survey, evenly distributed across the four demographics. Some of the responses to the survey are:

``There are always too many things to see \ldots I am wishing I can see more later'' (current student)

``Campus Preview Weekend (CPW) was really rushed, and I did not get to see everything'' (prospective freshman)

``I wish I could get a better sense of MIT culture. The tour guide briefly touched on how each dorm was different, but did not go into detail. '' (prospective freshman)

``I wish they had shown us where the students hang out '' (parent)

``I wish I could get a better sense of MIT culture. The tour guide briefly touched on how each dorm is different, but didn't go into detail. '' (tourist)

From the above responses, we saw that a common complaint was that people were not able to visit places they wanted to with the regular campus tours. In order to solve this problem, we proposed an application that helps create and share custom tours of the MIT campus called ``Campus Sherpa''.

Since smartphones offer location awareness and mobility, they are the obvious platform choice for ``Campus Sherpa''. By deploying this application for iOS, users can take out their smartphone when they arrive at MIT, select a tour based on their interest, and then begin exploring sites that will be most useful to them.

The second aspect of ``Campus Sherpa'' that we wished to support is to allow users to chronicle their tours and share them with others. As prospective students compare MIT to other schools or as tourists reminisce about their visit to MIT, the ability to record their tours (by associating locations with text, photos, videos, and links to related content) will aid them in remembering. For current students, this offers a way to record the precious memories they make at MIT and share them with others. Thus, the primary audience will be people who aim to tour MIT and record their experience there. The secondary audience will be current MIT students who would like to chronicle their lives at MIT.

While some MIT tours will be pre-installed, the majority should be user-created.
Based on our own experiences and the responses to our survey, people have varied tastes and desired destinations when visiting MIT, and we aimed to make sure that their trip to MIT is as enjoyable and productive as possible.

By creating an application to let users create, share, and follow custom tours of MIT, we hope to make college tours custom fit to each person’s taste, to make memories of MIT easy to record and share, and to make a platform for sharing tours.

\section{System Description}

``Campus Sherpa'' allows users to achieve two goals: 

\begin{itemize}
	\item Take custom tours
	\item Create tours based on experiences.
\end{itemize}
The app has a number of pages for different tasks, but the three most important tasks that users need to achieve are:

\begin{itemize}
	\item Make a Tour
	\item Select a Tour
	\item Take a Tour
	\item Make a Tour
\end{itemize}

We implement a tour as a set of geolocations (i.e. latitude and longitude), each associated with media (i.e. text, links, pictures, audio, video). The interface for making a tour consists of a screen which displays the tour so far - providing a map of the geolocations on the tour. Users can edit geolocations by selecting them on the map, or they can create a new one by clicking a button.

The geolocation creation/edit screen automatically fills in the user’s current location (via GPS) and a default geolocation name. The screen also displays the associated media for the geolocation and buttons to add new media (each of which has its own screen) (TODO: FIGURE FOR CREATE/EDIT)

Users should also be able to select a tour to take. The ``Select a Tour'' screen provides a list of the tours and a means for searching and filtering them. (TODO: FIGURE FOR SELECT TOUR)

The ``Take a Tour'' screen allows users to take tours displays a map of the user’s current location and other locations on the tour. The bottom of the screen includes a set of boxes with thumbnails for the associated media. When the user taps a box, the media is displayed on the screen (TODO: FIGURE FOR TAKE A TOUR)

There are other screens in ``Campus Sherpa'' - a home screen (from which we will launch other screens), a user profile screen, a settings screen, a help screen, etc. (TODO: SOME MISC FIGURE)

\subsection{Technical Details}
We use Parse for the backend. Parse is a backend-as-a-service which provides an SDK for iOS which allows users to persist objects on the server. This is much simpler than building a backend from scratch. For this project, we persist and retrieve latitude/longitude, text, pictures, and audio - which Parse supports. Since this application does not run in a browser, we do not have any browser dependencies (TODO: EXPAND THIS)
    
\section{Field Study}

On each page your material (not including the page number) should fit
within a rectangle of 18 x 23.5 cm (7 x 9.25 in.), centered on a US
letter page, beginning 1.9 cm (.75 in.) from the top of the page, with
a .85 cm (.33 in.) space between two 8.4 cm (3.3 in.) columns.  Right
margins should be justified, not ragged. Beware, especially when using
this template on a Macintosh, Word can change these dimensions in
unexpected ways. Please be sure that your PDF is US letter and not
A4. If your PDF or paper are formatted for A4, the submission will be
returned to you to fix.

\section{Discussion}

Prepare your submissions on a word processor or typesetter.  Please
note that page layout may change slightly depending upon the printer
you have specified.  \LaTeX\ sometimes will create overfull lines
that extend into columns.  To attempt to combat this, the .cls
file has a command, {\textbackslash}sloppy, that essentially asks
\LaTeX\ to prefer underfull lines with extra whitespace.  For more
details on this, and info on how to control it more finely, check out
{\url{http://www.economics.utoronto.ca/osborne/latex/PMAKEUP.HTM}}.

\subsection{Future Work}

Your paper's title, authors and affiliations should run across the
full width of the page in a single column 17.8 cm (7 in.) wide.  The
title should be in Helvetica 18-point bold; use Arial if Helvetica is
not available.  Authors' names should be in Times Roman 12-point bold,
and affiliations in Times Roman 12-point.  For more than three authors,
you may have to place some address information in a footnote, or in a named
section at the end of your paper. Please use full international addresses and
telephone dialing prefixes.  Leave one 10-pt line of white space below the last
line of affiliations.

\subsection{References}

Every submission should begin with an abstract of about 150 words,
followed by a set of keywords. The abstract and keywords should be
placed in the left column of the first page under the left half of the
title. The abstract should be a concise statement of the problem,
approach and conclusions of the work described.  It should clearly
state the paper's contribution to the field of HCI.

The first set of keywords will be used to index the paper in the
proceedings. The second set are used to catalogue the paper in the ACM
Digital Library. The latter are entries from the ACM Classification
System~\cite{acm_categories}.  In general, it should only be necessary
to pick one or more of the H5 subcategories, see
\url{http://www.acm.org/class/1998/ccs98.html}

\subsection{Conslusion}

Please use a 10-point Times Roman font or, if this is unavailable,
another proportional font with serifs, as close as possible in
appearance to Times Roman 10-point. The Press 10-point font available
to users of Script is a good substitute for Times Roman. If Times
Roman is not available, try the font named Computer Modern Roman. On a
Macintosh, use the font named Times and not Times New Roman. Please
use sans-serif or non-proportional fonts only for special purposes,
such as headings or source code text.

\section{Acknowledgments}

We thank the participants of our studies and exercises for their time and effort, as well as for allowing us
to use their thoughts and comments in both this paper, and in the development process of Campus Sherpa. \medskip

We also thank Ed Barrett, Frank Bentley, and Jason Martin Lipshin and the entire 21W.789 staff for their continued support and advice. Without them, Campus Sherpa would not have been possible.

% Balancing columns in a ref list is a bit of a pain because you
% either use a hack like flushend or balance, or manually insert
% a column break.  http://www.tex.ac.uk/cgi-bin/texfaq2html?label=balance
% multicols doesn't work because we're already in two-column mode,
% and flushend isn't awesome, so I choose balance.  See this
% for more info: http://cs.brown.edu/system/software/latex/doc/balance.pdf
%
% Note that in a perfect world balance wants to be in the first
% column of the last page.
%
% If balance doesn't work for you, you can remove that and
% hard-code a column break into the bbl file right before you
% submit:
%
% http://stackoverflow.com/questions/2149854/how-to-manually-equalize-columns-
% in-an-ieee-paper-if-using-bibtex
%
% Or, just remove \balance and give up on balancing the last page.
%
\balance

\section{References format}
References must be the same font size as other body text.
% REFERENCES FORMAT
% References must be the same font size as other body text.

\bibliographystyle{acm-sigchi}
\bibliography{sample}
\end{document}
